%%%%%%%%%%%%%%%%%%%%%%%%%%%%%%%%%%%%%%%%%
% fphw Assignment
% LaTeX Template
% Version 1.0 (27/04/2019)
%
% This template originates from:
% https://www.LaTeXTemplates.com
%
% Authors:
% Class by Felipe Portales-Oliva (f.portales.oliva@gmail.com) with template 
% content and modifications by Vel (vel@LaTeXTemplates.com)
%
% Template (this file) License:
% CC BY-NC-SA 3.0 (http://creativecommons.org/licenses/by-nc-sa/3.0/)
%
%%%%%%%%%%%%%%%%%%%%%%%%%%%%%%%%%%%%%%%%%

%----------------------------------------------------------------------------------------
%	PACKAGES AND OTHER DOCUMENT CONFIGURATIONS
%----------------------------------------------------------------------------------------

\documentclass[
	12pt, % Default font size, values between 10pt-12pt are allowed
	%letterpaper, % Uncomment for US letter paper size
	%spanish, % Uncomment for Spanish
]{fphw}

% Template-specific packages
\usepackage[utf8]{inputenc} % Required for inputting international characters

\usepackage[T1]{fontenc} % Output font encoding for international characters

\usepackage{mathpazo} % Use the Palatino font

\usepackage{xcolor}

\usepackage{graphicx} % Required for including images

\usepackage{booktabs} % Required for better horizontal rules in tables

\usepackage{listings} % Required for insertion of code

\usepackage{enumerate} % To modify the enumerate environment

%----------------------------------------------------------------------------------------
%	ASSIGNMENT INFORMATION
%----------------------------------------------------------------------------------------



\title{\textcolor{red}{Assignment 1}} % Change color to light red
\author{\textcolor{red}{ALI SARABANDI}} % Change color to light red
\date{\textcolor{red}{April 17th, 2023}} % Change color to light red
\institute{\textcolor{red}{University of Luigi Vanvitelli \\ Dep. of Mathematics}} % Change color to light red
\class{\textcolor{red}{Num. methodes}} % Change color to light red
\professor{\textcolor{red}{Prof. ROSANNA CAMPAGNA}} % Change color to light red

\begin{document}

\begin{center}
    \includegraphics[width=0.5\textwidth]{pngegg.png} 
\end{center}

%----------------------------------------------------------------------------------------

\begin{document}

\maketitle 

\section*{\textcolor{red}{\textbf{\emph{Question 1(A)}}}}


\begin{problem}
	
1. Download from Blackboard the file ModelReductionData.mat. The file contains a data matrix X ∈ R6×4000. Each column of X represents a data vector in the six dimensional space.


(a) Visualize the raw data using scatter plots, that is, select two components of the data at a time and plot them one against the other. You have 6 different two-dimensional
2

\end{problem}

%------------------------------------------------

\section*{\textcolor{red}{\textbf{\emph{Answer}}}}


The code loads the data from the file "ModelReductionData.mat" and creates 2D scatter plots for all possible pairs of the six components of the data. Each data point in the scatter plot represents a data vector in the six-dimensional space, with one component on the x-axis and another component on the y-axis. as you can see there is two cluster in most of the scatterplots. Based on the plots, it's quite apparent that certain correlations are much stronger than others. The unique aspect of these plots is that in some graphs, we can easily identify distinct straight lines, indicating that the variables are significantly positively or negatively correlated. However, in other graphs, we can observe two clusters forming two distinct groups of data points, as seen in Plot 7 or Plot 3.


\begin{figure}[htbp]
\centering
\includegraphics[width=0.3\textwidth]{1.jpg}
\hfill
\includegraphics[width=0.3\textwidth]{2.jpg}
\hfill
\includegraphics[width=0.3\textwidth]{3.jpg}

\vspace{0.5cm}

\includegraphics[width=0.3\textwidth]{4.jpg}
\hfill
\includegraphics[width=0.3\textwidth]{5.jpg}
\hfill
\includegraphics[width=0.3\textwidth]{6.jpg}

\vspace{0.5cm}

\includegraphics[width=0.3\textwidth]{7.jpg}
\hfill
\includegraphics[width=0.3\textwidth]{8.jpg}
\hfill
\includegraphics[width=0.3\textwidth]{9.jpg}

\vspace{0.5cm}

\includegraphics[width=0.3\textwidth]{10.jpg}
\hfill
\includegraphics[width=0.3\textwidth]{11.jpg}
\hfill
\includegraphics[width=0.3\textwidth]{12.jpg}

\vspace{0.5cm}

\includegraphics[width=0.3\textwidth]{13.jpg}
\hfill
\includegraphics[width=0.3\textwidth]{14.jpg}
\hfill
\includegraphics[width=0.3\textwidth]{15.jpg}

\caption{Scatter plots}
\end{figure}


%----------------------------------------------------------------------------------------



\newpage


\documentclass{article}
\usepackage{listings}
\lstset{
basicstyle=\ttfamily\small,
breaklines=true,
columns=fullflexible,
frame=single
}

\begin{lstlisting}[caption={My Matlab Code 1A}]

% Load data
load ModelReductionData.mat

% Plot 2D projections of data
for i = 1:6
    for j = i+1:6
        % Create a new figure for each pair of components
        figure()
        
        % Create a scatter plot with X(i,:) on the x-axis and X(j,:) on the y-axis
        scatter(X(i,:), X(j,:), 7, 'k', 'filled')
        
        % Label the x-axis and y-axis with the component numbers
        xlabel(['Component ', num2str(i)])
        ylabel(['Component ', num2str(j)])
        
        % Set the title of the plot with the component numbers
        title(['Scatter plot of Component ', num2str(i), ' vs. Component ', num2str(j)])
        
        % Set the axis to be equal and set the font size of the axis labels and title
        axis equal
        set(gca, 'FontSize', 20)
    end
end


\end{lstlisting}

	


\newpage

\section*{\textcolor{red}{\textbf{\emph{Question 1(B)}}}}

\begin{problem}
1.B Center the data and compute the SVD. Plot the singular values. What can you say about the dimensionality of the data? Show the scatter plots of the first few principal components. Do the plots suggest a presence of clusters in the data?
	

\end{problem}



\documentclass{article}
\usepackage{listings}
\lstset{
basicstyle=\ttfamily\small,
breaklines=true,
columns=fullflexible,
frame=single
}

\begin{lstlisting}[caption={My Matlab Code 1B}]

% Load the data
load('ModelReductionData.mat');

% Center the data by subtracting the mean of each row (dimension) from each data point
X_centered = X - mean(X,2);

% Compute the SVD of the centered data
[U,S,V] = svd(X_centered);

% Plot singular values

sing_values = diag(S);

plot(sing_values,'Ko','LineWidth',3,'MarkerSize',10)

set(gca,'FontSize',12);

xlabel('Principal Components')

ylabel('Singular Values')

title('Behaviour of singular values')

grid on

% Scatterplot of first two principal components

Z = U(:,1:2)' * X_centered;

figure(2)

plot(Z(1,:),Z(2,:),'k.','MarkerSize',3)

set(gca,'FontSize',18)

\end{lstlisting}


\subsection*{\textcolor{red}{\textbf{\emph{ Description of first part of the question}}}}


As you can see the codes above, we first load the data from the file "ModelReductionData.mat". We then centre the data by subtracting the mean of each row from each data point. Next, we compute the SVD of the centred data using the MATLAB built-in function "SVD". We extract the singular values from the diagonal matrix S and plot them using the MATLAB function "plot". Finally, we set the font size, add labels and a title to the plot, and turn on the grid. 

\subsection*{\textcolor{red}{\textbf{\emph{ Description of second part of the question}}}}

Based on the plot of the singular values, we can see that there is a significant drop in magnitude after the first two singular values, which suggests that the data may be effectively represented in a lower dimensional space. This is also supported by the fact that the first two principal components explain a large portion of the variance in the data, as seen in the scatter plots which you can find at the end of this section.

\subsection*{\textcolor{red}{\textbf{\emph{ Description of third part of the question}}}}

To plot the scatter plots of the first few principal components, we first need to compute the principal components, as you can see in the scatter plot we have 2 clusters. 

\begin{figure}[h!]
    \centering
    \includegraphics[width=0.6\textwidth]{plot1.jpg}
    \includegraphics[width=0.6\textwidth]{plot2.jpg}
      \label{scatterplots}
\end{figure}

\newpage


\section*{\textcolor{red}{\textbf{\emph{Question 2}}}}

\begin{problem}

Download from Blackboard the data file HandwrittedDigits.mat, containing the data matrix X of size 256 × 1 707 containing pixel images of the handwritten digits, 
and the label vector I of length 1707 containing numbers from 0 to 9, indicating the digits that the corresponding images represent.
Extract from X the images that correspond to numbers 0, 1, 3 and 7. 
From each subgroup, select 5 samples, and approximate these samples by the linear combination of the first k feature vectors, k = 5, 10, 15, 20, 25, that is,
k
x(j) ≈ Pkx(j) = X z(j)u(l),
l l=1
where z(j) are the principal components of x(j). Plot the approximation as images, as well l
as the residual, x(j) − Pkx(j). Plot the norms of the errors as a function of k.


\end{problem}



documentclass{article}
\usepackage{listings}
\lstset{
basicstyle=\ttfamily\small,
breaklines=true,
columns=fullflexible,
frame=single
}

\begin{lstlisting}[caption={My Matlab Code 2}]

clear all
close all
load('HandwrittenDigits.mat');


% Extract images for digits 0, 1, 3, and 7
digit_indices = find(I == 0 | I == 1 | I == 3 | I == 7);
digit_images = X(:, digit_indices);

% Select 5 samples from each subgroup
num_samples = 5;
samples = [];
for digit = [0, 1, 3, 7]
    digit_indices = find(I == digit);
    digit_images = X(:, digit_indices);
    samples = digit_images(:,1:5);


% Compute principal components of X
[U, ~, ~] = svd(digit_images);

% Compute approximations for each sample and plot results
ks = [5, 10, 15, 20, 25];
for i = 1:size(samples, 2)
    figure;
    for j = 1:length(ks)
        k = ks(j);
        % Compute approximation using k principal components
        z = U(:, 1:k)' * samples(:, i);
        approx = U(:, 1:k) * z;
        
        % Compute residual
        residual = samples(:, i) - approx;
        
        % Plot approximation and residual
        subplot(2, length(ks), j);
        imshow(reshape(approx, [16, 16])', []);
        title(sprintf('k = %d', k));
        
        subplot(2, length(ks), j + length(ks));
        imshow(reshape(residual, [16, 16])', []);
    end
end
end
% Initialize vector to store norms of errors
error_norms = zeros(length(ks), size(samples, 2));

% Compute errors for each sample and store norms
for i = 1:size(samples, 2)
    for j = 1:length(ks)
        k = ks(j);
        % Compute approximation using k principal components
        z = U(:, 1:k)' * samples(:, i);
        approx = U(:, 1:k) * z;
        
        % Compute residual and store norm
        residual = samples(:, i) - approx;
        error_norms(j, i) = norm(residual);
    end
end

% Plot norms of errors as a function of k
figure;
plot(ks, error_norms(:,1), '-o');
xlabel('k');
ylabel('Norm of error');
legend('residuo');

\end{lstlisting}



\begin{figure}[htbp]
\centering
\includegraphics[width=0.3\textwidth]{image1.jpg}
\hfill
\includegraphics[width=0.3\textwidth]{image2.jpg}
\hfill
\includegraphics[width=0.3\textwidth]{image3.jpg}
\end{figure}

\begin{figure}[htbp]
\centering
\includegraphics[width=0.3\textwidth]{image4.jpg}
\hfill
\includegraphics[width=0.3\textwidth]{image5.jpg}
\hfill
\includegraphics[width=0.3\textwidth]{image6.jpg}
\end{figure}

\begin{figure}[htbp]
\centering
\includegraphics[width=0.3\textwidth]{image7.jpg}
\hfill
\includegraphics[width=0.3\textwidth]{image8.jpg}
\hfill
\includegraphics[width=0.3\textwidth]{image9.jpg}
\end{figure}

\begin{figure}[htbp]
\centering
\includegraphics[width=0.3\textwidth]{image10.jpg}
\hfill
\includegraphics[width=0.3\textwidth]{image11.jpg}
\hfill
\includegraphics[width=0.3\textwidth]{image12.jpg}
\end{figure}

\begin{figure}[htbp]
\centering
\includegraphics[width=0.3\textwidth]{image13.jpg}
\hfill
\includegraphics[width=0.3\textwidth]{image14.jpg}
\hfill
\includegraphics[width=0.3\textwidth]{image15.jpg}
\end{figure}

\clearpage


\begin{figure}[htbp]
\centering
\includegraphics[width=0.3\textwidth]{image16.jpg}
\hfill
\includegraphics[width=0.3\textwidth]{image17.jpg}
\hfill
\includegraphics[width=0.3\textwidth]{image18.jpg}
\end{figure}

\begin{figure}[htbp]
\centering
\includegraphics[width=0.3\textwidth]{image19.jpg}
\hfill
\includegraphics[width=0.3\textwidth]{image20.jpg}
\end{figure}

\begin{figure}[htbp]
\centering
\includegraphics[width=0.7\columnwidth]{normerror1.jpg}
\end{figure}


\subsection*{\textcolor{red}{\textbf{\emph{ Description of first part of the question}}}}
The results of this code show the approximation of each image using a varying number of principal components. As k increases, the approximation becomes more accurate and visible and the residual becomes smaller. However, it's important to note that increasing k also increases the complexity of the model, and there may be a point where the additional complexity does not provide a significant improvement in approximation accuracy.

\subsection*{\textcolor{red}{\textbf{\emph{ Description of second part of the question}}}}
The plot now shows only one line for each sample but you can change the number to see the plots for the others or you can just delete the selection part and get a plot for all the line , but in this case makes it easier to compare the performance of the different samples. We can also see that the error decreases first for all samples, as k increases, which is consistent with the interpretation from before. 

\subsection*{\textcolor{red}{\textbf{\emph{conclusion}}}}
Each plot in the figure represents one of the 5 vectors of the chosen digits: 0, 1, 3, and 7. For each sample, it is approximated using k feature vectors ranging from 5 to 25, incremented by 5 each time. To illustrate, let's take the example of digit 7: the first subplot represents the approximation with 5 feature vectors, the second with 10, the third with 15, the fourth with 20, and the fifth with 25 (on the first page, we have the subplot approximated with 25 feature vectors). On the second line of each image, the residuals are plotted, approximated with the feature vectors precisely as before. The residuals refer to the difference between the approximation and the starting image. The main objective of these plots is to demonstrate that as the number of feature vectors increases, the image becomes more distinct and well-defined. This is because the error decreases as the number of feature vectors increases. 


\section*{\textcolor{red}{\textbf{\emph{Question 3}}}}

\begin{problem}


Download from Blackboard the data file IrisData.mat. The matrix X consists of 150 vectors, each one having four components. The data correspond to measurements of certain dimensions in three species of flowers, Iris setosa, Iris versicolor, and Iris virginica, and the components of the data vectors. By using the PCA, investigate if the data set suggests the presence of clusters that would make it possible to separate the three species from each other. (Later on, the flower species corresponding to each data point will be made available.)	

\end{problem}

\subsection*{\textcolor{red}{\textbf{\emph{ Answer}}}}

The plot suggests that there are three distinct clusters in the data that correspond to the three different species of flowers . Therefore, it is possible to separate the species from each other using the first two principal components. However, it is important to note it is not so visible that we have two cluster at the right side of the plot but if you color them it is much more visible that we have three cluster which each of them correspond to a species.

\subsection*{\textcolor{red}{\textbf{\emph{ conclusion}}}}
From the plot below, it is apparent that one cluster on the left is easily identifiable, whereas the same cannot be said for the points situated on the right of the plot, which form another group of points. This suggests that the characteristics used to describe the iris species may not be sufficient to distinguish between all three species. Therefore, the primary conclusion we can draw is that, based on the graph, we can mostly identify two clusters instead of three.

\begin{center}
    \includegraphics[width=0.5\columnwidth]{last1.jpg} 
\end{center}

documentclass{article}
\usepackage{listings}
\lstset{
basicstyle=\ttfamily\small,
breaklines=true,
columns=fullflexible,
frame=single
}

\begin{lstlisting}[caption={My Matlab Code 2}]

close all
clear all
load('IrisDataAnnotated.mat');

% Center the data
p = size(X,2);
X_mean = 1/p * sum(X,2);
X_centered= X - X_mean * ones(1,p);


% Perform PCA
[U, S, V] = svd(X_centered);

% Project the data onto the first two principal components
Z = U(:,1:2)' * X_centered ;

% Plot the data in the new 2D space
figure;
plot(Z(1,:), Z(2,:),'k.','MarkerSize',10);
xlabel('PC1');
ylabel('PC2');
title('Iris dataset projected onto the first two principal components');

\end{lstlisting}



\section*{\textcolor{blue}{\textbf{PCA project of Yale}}}


\begin{problem}

(a) Plot 5 different columns of the matrix Xr as gray scale images. Choose faces that you think are in some sense representative of different groups.

(b) After having computed the first singular values/vectors, plot the singular values and comment on how fast (or slowly) they decrease. Notice, however, that computing the full SVD may be very slow, so use svds, increasing gradually r. It may be more informative to plot their logarithms.

(c) Plot the first 5 feature vectors (columns of U) as gray scale images.

(d) Approximate the 5 images corresponding to the columns that you selected by a linear combination of the first k = 4; 8; 15 feature vectors with coefficients their principal components. Each time display the approximation and difference between it and the original data in the form of a gray scale image.

\end{problem}

\subsection*{\textcolor{blue}{\textbf{\emph{Explanation and Interpretation }}}}
      

Explanation:

(a) we load the Yale face dataset and define some variables for displaying the images. We then choose a rank for the low-rank approximation and compute the truncated SVD using svds. 
We then choose 5 different columns of Xr and display them as grey scale images using imshow.

(b) we again load the dataset and define some variables. We then loop over different ranks, computing the truncated SVD at each rank and extracting the singular values. 

(c) we again load the dataset and define some variables. We compute the truncated SVD and extract the first 5 columns of the left singular vectors U. 
We then display each of these feature vectors as gray scale images using imshow.

Interpretation:

(a) this part shows us what some representative faces look like after being approximated using a low-rank approximation. 
By choosing different columns of Xr, we can see how well different faces are represented by the low-rank approximation.

    \includegraphics[width=0.9\textwidth]{pca1.jpg} 

\hfill

(b) this part shows us how quickly the singular values decrease as we increase the rank of the truncated SVD. 
This is important because it tells us how many singular values we need to keep in order to get a good approximation of the data.

\begin{center}
  \includegraphics[width=0.7\textwidth]{pca3.jpg} 
  \includegraphics[width=0.7\textwidth]{pca2.jpg}
\end{center}

\hfill


(c) this part shows us what the first few feature vectors look like. These vectors are important because they tell us what directions in the data 
are most important for explaining the variability in the data. By visualizing these feature vectors, we can get an idea of what kinds of patterns are most important in the dataset.

\begin{center}
\includegraphics[width=0.7\textwidth]{pca4.jpg}
\end{center}

    
\hfill
    
    
 (d) I extracts the reconstructed face using the first k principal components and displays it in a grayscale image. I also computed the difference between the reconstructed face and the original image, which is stored as a residual image and also displayed in grayscale.
Overall, the code is performing dimensionality reduction and image compression by approximating the original images with a reduced set of principal components. By displaying the reconstructed faces and residuals for different values of k, the code shows how the quality of the approximation improves as more principal components are included.

       \includegraphics[width=0.9\textwidth]{d1.jpg} 

        \includegraphics[width=0.9\textwidth]{d2.jpg} 
        
        \includegraphics[width=0.9\textwidth]{d3.jpg} 
          
        \includegraphics[width=0.9\textwidth]{d4.jpg} 
        
        \includegraphics[width=0.9\textwidth]{d5.jpg} 


\hfill

  

    
\newpage    
    

\documentclass{article}
\usepackage{listings}
\lstset{
basicstyle=\ttfamily\small,
breaklines=true,
columns=fullflexible,
frame=single
}
\begin{document}

\begin{lstlisting}[language=Matlab,caption={My Matlab Code}]

load Yale_64x64.mat
faceW = 64;
faceH = 64;
numPerLine = 11;
ShowLine = 2;
% Display original faces
Y = zeros(faceH*ShowLine,faceW*numPerLine); 
for i=0:ShowLine-1
    for j=0:numPerLine-1 
        Y(i*faceH+1:(i+1)*faceH,j*faceW+1:(j+1)*faceW) = reshape(fea(i*numPerLine+j+1,:),[faceH,faceW]);
    end 
end
figure; imagesc(Y);colormap(gray); axis off;

% Principal Component Analysis
X = fea';
X_mean = mean(X,2);
X_cen = X - X_mean;
r = 10; % Choose as you wish
[U,D,V] = svds(X_cen,r);
X_r = U*D*V' + X_mean;

% Plot 5 different columns of the matrix X_r as gray scale images
figure;
cols = [10, 20, 30, 40, 50];
for i=1:length(cols)
    subplot(2,3,i); imshow(reshape(X_r(:,cols(i)),[faceH,faceW]),[]); title(['Face ',num2str(cols(i))]);
end

% Plot the singular values
figure;
s = diag(D).^2 / sum(diag(D).^2);
semilogy(s,'o-'); title('Singular Values'); xlabel('Singular Value Index'); ylabel('Singular Value');

% Plot the first 5 feature vectors (columns of U) as gray scale images
figure;
for i=1:5
    subplot(1,5,i); imshow(reshape(U(:,i),[],faceW),[]); title(['PC ',num2str(i)]);
end

% Load the Yale face
load Yale_64x64.mat
faceW = 64;
faceH = 64;
numPerLine = 3;
ShowLine = 2;

% Create a matrix Y to store the images to be displayed.
Y = zeros(faceH*ShowLine,faceW*numPerLine);

% Loop over faces.
for face = [2,15,28,41,54]
    % Initialize a counter for the number of truncated SVDs computed.
    ind = 0;
    % Create a new figure.
    figure()
    % Loop over ranks.
    for r = [4 8 15]
        % Compute the mean face.
        fea_means = sum(fea,2)/size(fea,2);
        % Center the data.
        fea_centered = fea - fea_means*ones(1,size(fea,2));
        % Compute the truncated SVD.
        [U,D,V] = svds(fea_centered,r);
        % Compute the approximation of face.
        fea_r = U*D*V';
        % Extract the reconstructed face and the residual from the approximation.
        reconstructed_face = reshape(fea_r(face,:),[faceW,faceH]);
        residual_face = reshape(fea_centered(face,:)-fea_r(face,:),[faceW,faceH]);
        % Store the images to be displayed in Y.
        Y(1:64,ind*faceW+1:(ind+1)*faceW) = reconstructed_face;
        Y(65:128,ind*faceW+1:(ind+1)*faceW) = residual_face;
        % Increment the counter.
        ind = ind + 1;
    end
    % Display the images stored in Y.
    imagesc(Y);
    colormap(gray);
end


\end{lstlisting}

\end{document}





  \end{document}

